\section{Introduction}

\GROOVE is a project centered around the use of simple graphs for
modelling the design-time, compile-time, and run-time structure of
object-oriented systems, and graph transformations as a basis for
model transformation and operational semantics. This entails a formal
foundation for  model transformation and dynamic semantics, and the
ability to verify model transformation and dynamic semantics through
an (automatic) analysis of the resulting graph transformation systems,
for instance using model checking.

This manual constsis of some download and installation instructions and a
manual for using the tools included in the \GROOVE tool set. The latter also
explains the format used for graphs and graph transformations. Together
with some examples, this should allow you to get started with \GROOVE.

\subsection{Toolkit Components}
\stlabel{toolkit}

The \GROOVE tool set includes the following programs:
\begin{description}\noitemsep
\item[Simulator:] a GUI-based tool that lets you construct, simulate and
  model check rule systems visually;
\item[Editor:] a GUI-based editor that lets you edit individual rules and
  graphs;
\item[Generator:] a command line tool that lets you simulate and model check
  rule systems without the performance penalty of the GUI;
\item[Imager:] a command line or GUI tool that suppots various conversions from
  \GROOVE graphs and rules to other visual formats.
\end{description}

\subsection{Getting it running}

Since the entire \GROOVE tool is written in Java, getting it running is
extremely easy.

\subsubsection{Download}

The \GROOVE tool is distributed under the Apache License, Version 2.0. A copy of
this license is available on \url{http://www.apache.org/licenses/LICENSE-2.0}.
The latest \GROOVE build can be downloaded from the \GROOVE sourceforge page:

\begin{center}
\url{http://sourceforge.net/projects/groove}
\end{center}

There are some different distributions of the \GROOVE tool set available on the
sourceforge site. The \emph{groove-bin+lib} package includes all the libraries
below. The \emph{groove-bin} package is identical but without the libraries.
The \emph{groove-src} package only includes the sources of the groove project.
There are also some examples available in the \emph{groove-samples} package.
The \emph{groove-doc} package consists of some publications about \GROOVE and
the theory behind it.


The \GROOVE library depends on some other libraries, namely:

\begin{itemize}\noitemsep
\item \textsc{Antlr}, for compiling control programs. \\
  See \url{http://www.antlr.org} (Version included: 3.4)
\item \textsc{Asm}, for Java bytecode manipulation and analysis (in conjunction
  with \textsc{Groovy}, see below). \\
  See \url{http://asm.ow2.org/} (Version included: 4.0)
\item \textsc{Gnu Prolog}, an implementation of ISO Prolog as a Java
  library. \\
  See \url{http://www.gnu.org/software/gnuprologjava/} (Version included:
  0.2.6)
\item \textsc{JGraph} for displaying graphs and rules. \\
  See \url{http://www.jgraph.com} (Version included: 5.13.0)
\item \textsc{EMF} for converting to and from Eclipse \textsc{ecore} format. \\
  See \url{http://eclipse.org} (version included: 2.5.0)
\item \textsc{EPSGraphics} for exporting displayed graps to EPS format. \\
  See \url{http://www.abeel.be/epsgraphics/} (Version included: 1.2)
\item \textsc{Groovy} for easy and flexible access to the \GROOVE API. \\
  See \url{http://groovy.codehaus.org/} (version included: 2.0.5)
\item \textsc{iText} for exporting displayed graphs to PDF format. \\
  See \url{http://itextpdf.com/} (Version included: 5.3.2)
\item \textsc{JGoodies Looks} for platform-dependent look\&feel. \\
  See \url{http://www.jgoodies.com/freeware/libraries/looks/} (Version
  included: 2.4.1)
\item \textsc{ltl2buchi}, a component to translate LTL formulae to
  B\"uchi automata. \\ 
  See \url{http://ti.arc.nasa.gov/profile/dimitra/projects-tools}
\item \textsc{RSyntaxTextArea}, for editing syntax-highlighted control
  programs. \\
  See \url{http://fifesoft.com/rsyntaxtextarea}
\end{itemize}

\subsubsection{Installation}

To use the \GROOVE tool set, download the bin+lib package from the download
site explained above and unzip it to any location on your computer. Let's refer
to this location as the \GROOVE directory.

The \texttt{bin} subdirectory of the \GROOVE directory contains \texttt{jar}
files for each of the toolkit programs (see \stref{toolkit}), so
\texttt{Simulator.jar}, \texttt{Editor.jar} etc. You can use these in either of
the following ways:
\begin{itemize}
\item In an explorer window opened on the \texttt{bin} directory, double-click
  the icon of the \texttt{jar} file;
\item On the command line, run %
  `\verb|java -jar GROOVE_PATH\bin\Program.jar [parameters]|', %
  where \texttt{GROOVE\_PATH} is the groove directory and \texttt{Program} is
  the toolkit program in question.
\end{itemize}
