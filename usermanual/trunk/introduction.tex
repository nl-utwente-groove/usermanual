\section{Introduction}

GROOVE is a project centered around the use of simple graphs for
modelling the design-time, compile-time, and run-time structure of
object-oriented systems, and graph transformations as a basis for
model transformation and operational semantics. This entails a formal
foundation for  model transformation and dynamic semantics, and the
ability to verify model transformation and dynamic semantics through
an (automatic) analysis of the resulting graph transformation systems,
for instance using model checking.

This manual constsis of some download and installation instructions and a
manual for using the tools included in the GROOVE tool set. The latter also
explains the format used for graphs and graph transformations. Together
with some examples, this should allow you to get started with GROOVE.

\subsection{Toolkit Components}

The GROOVE tool set includes an editor for creating graph production
rules, a simulator for visually computing the graph transformations
induced by a set of graph production rules, a generator for
automatically exploring state spaces, and an imaging tool for
converting graphs to images.

\subsection{Getting it running}

Since the entire \Groove{} tool is written in Java, getting it running is
extremely easy.

\subsubsection{Download}

The GROOVE tool is distributed under the Apache License, Version 2.0. A copy of
this license is available on \url{http://www.apache.org/licenses/LICENSE-2.0}.
The latest GROOVE build can be downloaded from the GROOVE sourceforge page:

\begin{center}
\url{http://sourceforge.net/projects/groove}
\end{center}

There are some different distributions of the GROOVE tool set available on the
sourceforge site. The \emph{groove-bin+lib} package includes all the libraries
below. The \emph{groove-bin} package is identical but without the libraries.
The \emph{groove-src} package only includes the sources of the groove project.
There are also some examples available in the \emph{groove-samples} package.
The \emph{groove-doc} package consists of some publications about GROOVE and
the theory behind it.


The GROOVE library depends on some other libraties, namely:

\begin{itemize}\noitemsep
\item Castor - \url{http://www.castor.org/} (Version included: 0.9.5.2)
\item JGraph - \url{http://www.jgraph.com/} (Version included: 5.9.2)
\item JGraphAddons (Version included: 1.0.4)
\item EPS Graphics Library - \url{http://sourceforge.net/projects/epsgraphics/} (Version included: 1.0.0)
\item Xerces - \url{http://xerces.apache.org/xerces2-j/} (Version included 2.6.0 (Impl))
\end{itemize}


\subsubsection{Installation on Windows systems}

To use the GROOVE tool set under windows, download the bin+lib package from the
download site explained above and unzip it to any location on your computer. We
will refer to the destination folder as GROOVE\_PATH. Under GROOVE\_PATH, you
will find a directory \emph{bin} which contains some batchfiles, that refer to
the tools contained by the GROOVE tool set (Editor.bat, Simulator.bat,
Generator.bat and Imager.bat). Inside the batchfiles, a variable
GROOVE\_BIN\_DIR is set. Make sure the value it is set to is your local
GROOVE\_PATH.

Then make sure you have a JRE (Java Runtime Environment) version 5.0 (or
higher) installed and that you have it in your PATH. You can download a JRE
from \url{http://java.sun.com}.

Now double-click on a batch-file to run the desired tool.

\subsubsection{Installation on UNIX based systems}

To use the GROOVE tool set under windows, download the bin+lib package from the
download site explained above and unzip it to any location on your computer. We
will refer to the destination folder as GROOVE\_PATH. Under GROOVE\_PATH, you
will find a directory \emph{bin} which contains some executable shell-scripts,
that refer to the tools contained by the GROOVE tool set (Editor, Simulator,
Generator and Imager). Inside the shell-scripts, a variable GROOVE\_BIN\_DIR is
set. Make sure the value it is set to is your local GROOVE\_PATH.

Then make sure you have a JRE (Java Runtime Environment) version 5.0 (or
higher) installed and that you have it in your PATH. You can download a JRE
from \url{http://java.sun.com}.

Now execute a shell-script to run the desired tool.
