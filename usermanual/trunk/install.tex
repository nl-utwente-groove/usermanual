\section{Getting started}

\subsection{Installation on Windows systems}

To use the GROOVE tool set under windows, download the bin+lib package from the download site explained above and unzip it to any location on your computer. We will refer to the destination folder as GROOVE\_PATH. Under GROOVE\_PATH, you will find a directory \emph{bin} which contains some batchfiles, that refer to the tools contained by the GROOVE tool set (Editor.bat, Simulator.bat, Generator.bat and Imager.bat). Inside the batchfiles, a variable GROOVE\_BIN\_DIR is set. Make sure the value it is set to is your local GROOVE\_PATH.

Then make sure you have a JRE (Java Runtime Environment) installed and that you have it in your PATH. You can download a JRE from \url{http://java.sun.com}.

Now double-click on a batch-file to run the desired tool.

\subsection{Installation on UNIX based systems}

To use the GROOVE tool set under windows, download the bin+lib package from the download site explained above and unzip it to any location on your computer. We will refer to the destination folder as GROOVE\_PATH. Under GROOVE\_PATH, you will find a directory \emph{bin} which contains some executable shell-scripts, that refer to the tools contained by the GROOVE tool set (Editor, Simulator, Generator and Imager). Inside the shell-scripts, a variable GROOVE\_BIN\_DIR is set. Make sure the value it is set to is your local GROOVE\_PATH.

Then make sure you have a JRE (Java Runtime Environment) installed and that you have it in your PATH. You can download a JRE from \url{http://java.sun.com}.

Now execute a shell-script to run the desired tool.

\subsection{Simulator}

The simulator is used to visually explore a grammar, the resulting transition system, and the states it contains.

\begin{verbatim}
Usage: Simulator <directory> [<start-state>]
\end{verbatim}

The Simulator can be started with two command-line options. First, the \emph{directory} containing the graph production system, and second, given a production system, the name of the start state that should be used. Without a start state given, the default start state \emph{start.gst} will be loaded.

Once the Simulator is started, a production system can also be loaded from the menu. Also, a start state other then the default start state can be loaded from the menu as well.

\subsection{Editor}

The editor can be used to edit state graphs and production rules. Command line options are shown below.

\begin{verbatim}
Usage: Editor <file>
\end{verbatim}

The optional file argument can contain either a graph (.gst) or a production rule (.gpr).

\subsection{Generator}

The generator will take a grammer, explore it, and store the resulting transition system. Without any options given, a gui will be shown where the available options can be selected. Command line usage is shown below.

\begin{verbatim}
Usage: Generator  [options] <grammar-location> [<start-graph-name>]

Options:
-o file Save the result to 'file' (GXL format, default extension .gxl)
-v val  Set the verbosity to 'val', in the range 0-2 (default = 1)
-l dir  Log the generation process, writing the file to the directory 'dir'
-x str  Set the exploration strategy. Legal values for 'str' are:
  depthfirst - Recursively calls itself on all reachable, non-explored states
  node-bounded - Only explores states where the node count does not exceed
    a given bound
  invariant:[!]<condition> - Stops exploring if the (negated) invariant 
    condition is violated
  full - At each pass, asks the LTS for all remaining open states, and 
    closes them
  barbed - Closes the state, picks an arbitrary outgoing transition and 
    continues with the target state if open
  edge-bounded - Only explores states where the edge counts do not exceed
    preset bounds
  live - Stops exploring if a final state is encountered
  linear - Generates a single outgoing transition and continues with the 
    target state
  bounded:[!]<condition> - Explores all states where the (negated) bounding 
    condition holds
  branching - At each pass, closes all open states; then continues with the 
    newly generated open states
\end{verbatim}

\subsection{Imager}

The imager can generate pictures of state graphs, production rules and transition systems.

\begin{verbatim}
Imager <input-file> <output-file>
\end{verbatim}

The Imager can take two arguments, namely the input-file and the output-file. The input-file can be a groove state graph (.gst), a groove production rule (.gpr) or a gxl file, which is created when a transition system is saved. The output-file can be either a .png or a .jpg file.
