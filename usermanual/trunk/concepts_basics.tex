\section{Basic Concepts}

When working in \Groove{}, in particular when creating or modifying rule
systems, you should reaslise that there are two different display modes for
graphs and rules: the one that you usually see in the Simulator, and the one of
the Editor. We call these the Display and Edit views, respectively.
Historically, the Edit view only exists because we never took the time to
create a better graph editor; instead, we ripped off a prototype editor
provided for free with the \JGraph{} library.

\subsection{Graphs}

\Groove{} uses edge-labelled, directed graphs. Graph nodes are depicted as
boxes, and edges as arrows between them. Self-edges (often also called loops,
i.e., edges whose source and target nodes coincide) are often not displayed as
arrows; instead, node labels are used to represent labels of self-edges. It is
important to realise that these are formally \emph{exactly the same}. For
instance, the left and right hand side of \fref{self-edges} depict exactly the
same graph.

\begin{figure}[h]
\begin{center}
\graphfig{self-edges-left}
\qquad
\graphfig{self-edges-right}
\end{center}
\caption{Node labels represent self-edges, hence these two views depict the
same graph.}
\flabel{self-edges}
\end{figure}

In the \Groove{} views, both nodes and edges can have multiple labels. Multiple
node labels are given through a vertical list; multiple edge labels are given
in the form of a comma-separated list. In both cases, these lists actually
represent multiple edges; for instance, the left and right hand sides of
\fref{multiple-edges} depict the same graph.

\begin{figure}[h]
\begin{center}
\graphfig{multiple-edges-left}
\qquad
\graphfig{multiple-edges-right}
\end{center}
\caption{Multiple labels represent multiple edges, hence these two views depict
the same graph.}
\flabel{multiple-edges}
\end{figure}

\subsection{Rules}

Formally, rules consist of left hand sides, right hand sides and negative
application conditions (NACs), all of which are different graphs, connected by
morphisms. In \Groove{} these graphs are combined into one single graphs, and
colour coding is used to distinguish the original components. Thus, a \Groove{}
view of a rule has the following kinds of elements:

\begin{description}
\item[Readers.] These are nodes and edges that are in both the LHS and the RHS.
  In both the editor and the display view, they are depicted just like ordinary
  graph elements; hence, the outlines are thin and black and the font colour is
  black.

\item[Erasers.] These are nodes and edges that occur in the LHS but not the
  RHS, meaning that they must be matched in order for the rule to apply, but by
  applying the rule they will be deleted. In the Display view, such elements
  are depicted by a thin, dashed blue outline and blue text. In the Edit view,
  erasers are distinguished by a special prefix ``\textsf{del:}''. For eraser
  nodes, this prefix should appear \emph{on its own} as a node label; for
  eraser edges, the prefix is followed by the edge label.

\item[Creators.] These are nodes and edges that occur in the RHS but not the
  LHS, meaning that they will be created when the rule is applied. In the
  Display view, such elements are depicted by a slightly wider, solid green
  outline (light grey in a black-and-white representation) and green text. In
  the Edit view, creators are distinguished by a special prefix
  ``\textsf{new:}''. For creator nodes, this prefix should appear \emph{on its
  own} as a node label; for creator edges, the prefix is followed by the edge
  label.

\item[Embargoes.] These are nodes and edges that are in a NAC, but not in the
  LHS. This means that they \emph{forbidden}: their presence in the host graph
  will prevent the rule from being applied. In the Display view, such elements
  are depicted by a wide, dashed red outline (darker grey in a black-and-white
  representation) and red text. In the Edit view, creators are distinguished by
  a special prefix ``\textsf{not:}''. For embargo nodes, this prefix should
  appear \emph{on its own} as a node label; for embargo edges, the prefix is
  followed by the edge label.
\end{description}
%
Another thing to note is that if a node plays any of the roles of eraser,
creator and embargo, its incident edges implicitly also have this role. Thus,
in that case the corresponding prefix can be omitted in the Edit view.

For example, \fref{simple-rule} shows the Edit and Display views of a rule
which contains all of these types of elements.

\begin{figure}[h]
\begin{center}
\graphfig{simple-rule-edit-view}
\qquad
\graphfig{simple-rule}
\end{center}
\caption{Display and Edit views of a simple rule.}
\flabel{simple-rule}
\end{figure}

\subsection{Negations}

Another way to forbid an edge is by inserting an exclamation mark in front of
its label. This therefore has the same effect as the ``\textsf{not:}'' prefix,
but it can only be used for edges. Moreover, negations can also be used
\emph{within} embargoes, achieving a double negation. For instance,
\fref{double-negation} expresses that the rule may only be applied if the
\textsf{Bus} has not already started (\textsf{!start}-self-edge), and there is
\emph{no} \textsf{Pupil} that is \emph{not} in the bus (\textsf{!in}-embargo
edge) --- in other words, if all the pupils are in the bus.

\viewsfig{double-negation}{
    Display and Edit views of a rule with duble negation.
}

Negation may only be osed on reader and embargo edges; in fact, they would be
meaningless when used on eraser or creator edges.

\subsection{Mergers and Injectivities}

\Groove{} has a special label ``\textsf{=}'' (the equals sign). When used
between nodes in a rule, this expresses that the nodes are really the same,
despite being depicted as different. Such equality labels may also be used on
creator edges (which are then called \emph{mergers}) and embargo edges (which
are then called \emph{injectitivies}). Moreover, they may be combined with
negation.

\paragraph{Mergers.}

\paragraph{Injectivities.}

In general, rules are not matched injectively --- meaning that distinct LHS
nodes may be matche dby the same host graph node. (See, however,
\stref{system-properties} where we discuss how to set a global injectivity
constraint through the system properties.)

\paragraph{Counting.}

As with ordinary labels, the effect of negation is in principle the same as
that of the embargo prefix --- but again, negations can be used \emph{within}
embargoes. This has an important use in enabling \emph{counting} in rules. For
instance, \fref{counting} specifies that a \textsf{Plate} may only be put in
the \textsf{Oven} if it contains \emph{exactly} three \textsf{Roll}s --- no
more and no less.

\viewsfig{counting}{Display and Edit view of a rule with a counting constraint}

\subsection{Priorities\by{IB}}

\subsection{Rule Comments\by{AR}}