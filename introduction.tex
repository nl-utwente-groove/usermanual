\section{Introduction}

GROOVE is a project centered around the use of simple graphs for
modelling the design-time, compile-time, and run-time structure of
object-oriented systems, and graph transformations as a basis for
model transformation and operational semantics. This entails a formal
foundation for  model transformation and dynamic semantics, and the
ability to verify model transformation and dynamic semantics through
an (automatic) analysis of the resulting graph transformation systems,
for instance using model checking.

The GROOVE tool set includes an editor for creating graph production
rules, a simulator for visually computing the graph transformations
induced by a set of graph production rules, a generator for
automatically exploring state spaces, and an imaging tool for
converting graphs to images.

This manual constsis of some download and installation instructions
and a manual for using the tools included in the groove tool set. The
latter also explains the format used for graphs and graph
transformations. Together with some examples, this should allow you to
get started with GROOVE.