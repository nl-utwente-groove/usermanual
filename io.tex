\section{I/O}
\stlabel{IO}

Graph grammars are stored in directories with extension ``\textsf{.gps}'' (for
\emph{g}raph \emph{p}roduction \emph{s}ystem). Each such directory contains all
information about one graph grammar, including rules, start state(s), control
program (if any) and system properties, in separate files.

\subsection{Graphs and rules}

The input/output format used by \Groove{} to store rules and graphs is GXL,
which is an XML format for interchange of graphs. See
\url{http://www.gupro.de/GXL/} for information about the format, including its
DTD and XSD; see also \cite{GXL}. There are some conventions in the way we have
used GXL to encode \Groove-specific information.

\paragraph{Edge labels.}

Edge labels are encoded as GXL edge attributes of type \textsf{String}. What is
actually stored is the full label, including prefixes, as seen in the Edit
view. Graphs and rules are stored in precisely the same fashion, except that
rules receive the extension ``\textsf{.gpr}'' (for \emph{g}raph
\emph{p}roduction \emph{r}ule) whereas graphs have extension ``\textsf{.gst}''
(for \emph{g}raph \emph{st}ate).

\paragraph{Graph attributes.}

We are using the GXL graph attributes to store some additional information
about the graphs, such as the version number of the I/O-format, the fact
whether it is a rule or a graph, and the priority and enabledness (if it is a
rule). For the latter two see \stref{rule-properties}.

\paragraph{Graph layout.}

For every GXL file, layout information is stored in a file whose name is that
of the GXL file, suffixed by the extension ``\textsf{.gl}''. For instance, the
layout of the file \textsf{rule.gpr} is contained in \textsf{rule.gpr.gl}.

Layout encoding is very ad hoc, and in fact based on the way the
graph-rendering library, \JGraph, stores this information internally (see
\cite{JGraph}). Every \textsf{gl}-file produced by \Groove{} contains some
comments briefly explaining the encoding.

\subsection{Control programs}
\stlabel{io-control}

Control programs are stored in plain text in files with extension
``\textsf{.gcp}'' (for \emph{g}raph \emph{c}ontorl \emph{p}rogram) within the
grammar directory. If there is a file \textsf{control.gcp}, this is taken to be
the default control program; others can be loaded in the Simulator.

\subsection{System properties}
\stlabel{io-system-properties}

System properties are stored in a file \textsf{system.properties} within the
grammar directory. The file is a plain text file, with lines of the form
``\textsf{resource = value}'' for the different properties discussed in
\stref{system-properties}.
