\begin{figure}
\begin{center}
\begin{tabular}{|l|c|p{105mm}|}
\hline\hline
Prefix & \sf Where? & Explanation \\
\hline
\remP & \sf NE,HRT & 
Remark node or edge; used for documentation purposes \\
\hline
\useP & \sf NE,R &
Declares a node or edge to be a reader (the default value) \\
\delP & \sf NE,R &
Declares a node or edge to be an eraser \\
\newP & \sf NE,R &
Declares a node or edge to be a creator \\
\cnewP & \sf NE,R &
Declares a node or edge to be a conditional creator \\
\notP & \sf NE,R &
Declares a node or edge to be an embargo \\
\hline
\attrP & \sf N,R &
Declares a node to be an (untyped) attribute value \\
\boolP & \sf NE,HRT &
On nodes, a boolean value or type; on edges, a boolean operator \\
\intP & \sf NE,HRT &
On nodes, an integer value or type; on edges, an integer operator \\
\realP & \sf NE,HRT &
On nodes, a real value or type; on edges, a real operator \\
\stringP & \sf NE,HRT &
On nodes, a string value or type; on edges, a string operator \\
\argP & \sf E,R &
Argument edge, from a product node to an attribute value \\
\prodP & \sf N,R &
Product node, collecting arguments for an attribute operation \\
\hline
\parP & \sf N,R &
Anonymous or numbered rule parameter node \\
\parinP & \sf N,R &
Numbered rule input parameter node \\
\paroutP & \sf N,R &
Numbered rule output parameter node \\
\hline
\absP & \sf NE,T &
Abstract type node or edge \\
\subP & \sf E,T &
Inheritance edge between node types \\
\hline
\forallP & \sf N,R &
Universal quantifier node \\
\forallxP & \sf N,R &
Non-vacuous universal quantifier node \\
\existsP & \sf N,R &
Existential quantifier node \\
\nestedP & \sf E,R &
Quantifier nesting edge, from rule node (\textsf{at})
or from quantifier node (\textsf{in}) \\
\hline
\idP & \sf N,R &
User-defined node identity \\
\hline
\colorP & \sf N,T &
Defines the text and outline colour of a node type \\
\hline
\pathP & \sf E,R &
Declares the remainder of the text to be a regular expression label \\
\litP & \sf NE,R &
Declares the remainder of the text to be a literal edge label \\
\typeP & \sf NE,HRT &
Declares the remainder of the text to be a node type \newline
May occur on edges only as part of a regular expression \\
\flagP & \sf NE,HRT &
Declares the remainder of the text to be a flag (= node label) \newline
May occur on edges only as part of a regular expression \\
\hline\hline
\end{tabular}
\end{center}
\caption{Overview of all edit prefixes. \newline 
``Where?'' values: \textsf{N}ode,
\textsf{E}dge, \textsf{G}raph, \textsf{R}ule, \textsf{T}ype graph}
\end{figure}


%%% Local Variables: 
%%% mode: latex
%%% TeX-master: "usermanual"
%%% End: 
