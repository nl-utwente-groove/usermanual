\title{User Manual for the GROOVE Tool Set}
\author{Arend Rensink, Iovka Boneva, Harmen Kastenberg and Tom Staijen \\
%
\vspace{0.2in} \\
%
Department of Computer Science, University of Twente \\
P.O.Box 217, 7500 AE Enschede, The Netherlands \\
\url{arend.rensink@utwente.nl}
}
\maketitle
%
\vspace{0.2in}

\clearpage
\tableofcontents

\clearpage
\subsection{Undocumented featores}

\GROOVE does not escape the ubiquitous problems of lagging documentation. In
the current version of this manual, the following features are poorly or not
at all documented:
%
\begin{itemize}[noitemsep]
\item Much improved attribute syntax (see \seccite{attributes}): essentially, you can now pretty much type expressions as you would expect.
\item Term algebra for attributes (see \seccite{attributes})
\item Counting quantifiers (see \seccite{nested})
\item Transition parameters can be set to \textsf{some} (see \seccite{system-properties})
\item Recipes: essentially atomically executed, named blocks, in many respects behaving just like rules (see \seccite{procedures})
\item Constraints: special, non-modifying rules that can be used to check for errors or enforced postconditions (see \seccite{properties})
\item Graph colouring and node-as-edge rendering (see \seccite{beauty})
\item Exploration strategies and their options (see \seccite{exploration}). 
\item Model checking capabilities (see \seccite{checking}). 
\item The import and export capabilities, including ECore compatibility (see \seccite{io}).
\end{itemize}

%%% Local Variables: 
%%% mode: latex
%%% TeX-master: usermanual
%%% End: 
