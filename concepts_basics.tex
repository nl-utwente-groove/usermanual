\section{Basic Concepts}
\stlabel{basic}

When working in \Groove{}, in particular when creating or modifying rule
systems, it is important to reaslise that there are two different display modes
for graphs and rules: the one that you usually see in the Simulator, and the
one of the Editor. We call these the Display and Edit views, respectively.
Historically, the Edit view only exists because we never took the time to
create a better graph editor; instead, we ripped off a prototype editor
provided for free with the \JGraph{} library.

\subsection{Graphs}
\stlabel{Graphs}

\Groove{} uses edge-labelled, directed graphs. Graph nodes are depicted as
boxes, and edges as arrows between them. Self-edges (often also called loops,
i.e., edges whose source and target nodes coincide) are often not displayed as
arrows; instead, node labels are used to represent labels of self-edges. It is
important to realise that these are formally \emph{exactly the same}. For
instance, the left and right hand side of \fref{self-edges} depict exactly the
same graph. (See, however, \stref{node types} below.)

\begin{figure}[h]
\begin{center}
%\graphbox{self-edges-left}
\tikzbox{self-edges-left}
\qquad
%\graphbox{self-edges-right}
\tikzbox{self-edges-right}
\end{center}
\caption{Node labels represent self-edges, hence these two views depict the
same graph.}
\flabel{self-edges}
\end{figure}

In the \Groove{} views, both nodes and edges can have multiple labels. Multiple
node labels are given through a vertical list; multiple edge labels are given
in the form of a comma-separated list. In both cases, these lists actually
represent multiple edges; for instance, the left and right hand sides of
\fref{multiple-edges} depict the same graph.

\begin{figure}[h]
\begin{center}
\graphbox{multiple-edges-left}
\qquad
\graphbox{multiple-edges-right}
\end{center}
\caption{Multiple labels represent multiple edges, hence these two views depict
the same graph.}
\flabel{multiple-edges}
\end{figure}

\paragraph{Labels in graphs.}

There are two restrictions on the character sequences that can be used as
labels in graphs:
\begin{itemize}\noitemsep
\item Colons (``\textsf{:}'') may not be used arbitrarily. This is because
  colons are used to separate \emph{label prefixes} (see below) from the
  remainder of the label. This restriction can be circumvented by
  \emph{starting} the label with a colon: this initial colon is then not
  counted to be part of the label proper, and the remainder is not parsed for
  colons. (Thus, an initial colon serves as an ``escape'' character precisely
  as an initial single quote serves as an escape in Excel.)
\item Whitespace other than simple spaces, such as tabs and newlines, cannot be
  included in labels.
\end{itemize}

\subsection{Node type labels}
\stlabel{node types}

Although, as stated above, node labels are actually self-edges, \Groove has a
special convention to indicate that certain edge labels may \emph{only} be used
for self-edges: namely, by preceding the label with the special prefix
``\textsf{type:}''. Such labels are intended to mimic node types, and in fact
in almost all respects they do behave like node types. For instance, every node
may have at most one ``\textsf{type:}''-prefixed self-edge; moreover, node type
labels are always displayed as the topmost label in the node (in case there are
more), and they are typeset in bold.

\viewsfig[tikz]{node-types}{Edit and Display views of a graph with node types}

For instance, \fref{node-types} shows two nodes with node type labels
\textsf{Library} and \textsf{Book}, respectively. The \textsf{Book}-node has
another self-edge \textsf{reserved}, also shown as a node label.

\subsection{Rules}
\stlabel{rules}

Formally, rules consist of left hand sides, right hand sides and negative
application conditions (NACs), all of which are different graphs, connected by
morphisms. In \Groove{} these graphs are combined into one single graph, and
colour coding is used to distinguish the original components. As a consequence,
a \Groove{} view of a rule has the following kinds of elements:

\begin{description}
\item[Readers.] These are nodes and edges that are in both the LHS and the RHS.
  In both the editor and the display view, they are depicted just like ordinary
  graph elements; hence, the outlines are thin and black and the font colour is
  black.

\item[Erasers.] These are nodes and edges that occur in the LHS but not the
  RHS, meaning that they must be matched in order for the rule to apply, but by
  applying the rule they will be deleted. In the Display view, such elements
  are depicted by a thin, dashed blue outline and blue text. In the Edit view,
  erasers are distinguished by a special prefix ``\textsf{del:}''. For eraser
  nodes, this prefix should appear \emph{on its own} as a node label; for
  eraser edges, the prefix is followed by the edge label.

\item[Creators.] These are nodes and edges that occur in the RHS but not the
  LHS, meaning that they will be created when the rule is applied. In the
  Display view, such elements are depicted by a slightly wider, solid green
  outline (light grey in a black-and-white representation) and green text. In
  the Edit view, creators are distinguished by a special prefix
  ``\textsf{new:}''. For creator nodes, this prefix should appear \emph{on its
  own} as a node label; for creator edges, the prefix is followed by the edge
  label.

\item[Embargoes.] These are nodes and edges that are in a NAC, but not in the
  LHS. This means that they \emph{forbidden}: their presence in the host graph
  will prevent the rule from being applied. In the Display view, such elements
  are depicted by a wide, dashed red outline (darker grey in a black-and-white
  representation) and red text. In the Edit view, creators are distinguished by
  a special prefix ``\textsf{not:}''. For embargo nodes, this prefix should
  appear \emph{on its own} as a node label; for embargo edges, the prefix is
  followed by the edge label.
\end{description}
%
Another thing to note is that if a node plays any of the roles of eraser,
creator and embargo, its incident edges implicitly also have this role. Thus,
in that case the corresponding prefix can be omitted in the Edit view.

For example, \fref{simple-rule} shows the Edit and Display views of a rule
which contains all of these types of elements.

\viewsfig{simple-rule}{%
     Edit and Display views of a simple rule.
}

\paragraph{Labels in rules.}

Label parsing in rules is more complicated than in graphs, because there are
many more special labels (see below for a discussion). The following points
should be noted.
%
\begin{itemize}\noitemsep
\item The rule for the use of colons is the same as for graphs: when an
  (unquoted) colon is used as part of a label, there should be a single initial
  colon preceding the entire label; this initial colon is not considered to be
  part of the label itself.

\item In addition to the above, whenever the spacial characters \textsf{'}
  (single quote), \textsf{\textbackslash} (backslash), \textsf{?} (question
  mark), \textsf{!}  (exclamation mark), \textsf{=} (equality sign), or
  \textsf{\{} and \textsf{\}} (opening and closing curly braces) are used
  literally within labels, i.e., not in their role as special characters, the
  whole label must be single-quoted. The surrounding single quotes are
  themselves not considered to be part of the label.

\item The backslash (``\textsf{\textbackslash}'') serves as an escape character
  within a single-quoted label: any next character (including the backslash
  itself) is interpreted literally rather than as a special character. This is
  especially needed to use single quotes within single-quoted labels.
\end{itemize}
%
For instance, the label \textsf{'\textbackslash\textbackslash?\textbackslash''}
(ending on two single quotes) in a rule matches the label
\textsf{\textbackslash?'} in a graph.

\paragraph{Rule names.}

Rules have names. The names are essentially identifiers. The actual constraints
on rule names are quite flexible: any string that can be used as a file name
but does not contain spaces or periods is allowed as a rule name. However, it
is \emph{recommended} to stick to rule names that are valid identifiers:
%
\begin{itemize}\noitemsep
\item Start a rule name with a letter --- by convention a small letter;
\item Restrict the remaining characters to letters, digits, underscores or
  dollar characters.
\end{itemize}
%
Rule names can impose a hierarchical structure, similar to the package
structure of qualified Java class names. For instance, the name
``$\textsf{a.b}$'' stands for rule ``\textsf{b}'' in package
``\textsf{a}''. This mechanism is only there for the purpose of structuring
larger sets of rules; the structure does not change the meaning of the rule
system (see also \stref{grammars} below).

\paragraph{Example usage.}

The use of the above features is demonstrated by the following \Groove samples:
%
\begin{itemize}\noitemsep
\item \textsf{circular-buffer}, a simple data structure with two rules,
  containing creators, erasers and embargoes;
\item \textsf{loose-nodes}, showing that node labels are just self-edges which
  can be added to existing, non-labelled nodes.
\end{itemize}

\subsection{Negations}
\stlabel{negation}

Another way to forbid an edge is by inserting an exclamation mark in front of
its label. This therefore has the same effect as the ``\textsf{not:}'' prefix,
but it can only be used for edges. Moreover, negations can also be used
\emph{within} embargoes, achieving a double negation. For instance,
\fref{double-negation} expresses that the rule may only be applied if the
\textsf{Bus} has not already started (\textsf{!start}-self-edge), and there is
\emph{no} \textsf{Pupil} that is \emph{not} in the bus (\textsf{!in}-embargo
edge) --- in other words, if all the pupils are in the bus.

\viewsfig{double-negation}{
     Edit and Display views of a rule with double negation.
}

Negations may only be used on reader and embargo edges; in fact, they would be
meaningless when used on eraser or creator edges.

\subsection{Equalities, Mergers and Injectivities}
\stlabel{equality}

\Groove{} has a special label ``\textsf{=}'' (the equals sign). When used
between nodes in a rule, this expresses that the nodes are really the same,
despite being depicted as different. Such equality labels may also be used on
creator edges (which are then called \emph{mergers}) and embargo edges (which
are then called \emph{injectitivies}). Moreover, they may be combined with
negation.

\paragraph{Mergers.}

\Groove{} rules can \emph{merge} nodes. This is specified by a special edge
labelled ``\textsf{new:=}'' between the nodes that are to be merged. The
direction of the edge is irrelevant. When two nodes are merged, the resulting
node receives the incident edges of both original nodes (including the
self-edges). For instance, the rule in \fref{merger} specifies that the start
and final state of an automaton should be merged, while all incoming and
outgoing transitions are preserved.

\viewsfig{merger}{%
     Edit and Display views of a rule with a merger.
}

\paragraph{Injectivities.}

In general, rules are not matched injectively --- meaning that distinct LHS
nodes may be matche dby the same host graph node. (See, however,
\stref{system-properties} where we discuss how to set a global injectivity
constraint through the system properties.) Local injectivity can be enforced by
a special edge labelled ``\textsf{!=}'' or ``\textsf{not:=}:'' the end nodes of
such an edge will always have distinct images. Note that the direction of the
edge is irrelevant. For instance, the rule in \fref{injectivity} specifies that
a couple may only marry if they do not share parents.

\viewsfig{injectivity}{%
    Edit and Display views of a rule with two injectivity constraints.
}

\paragraph{Counting.}

As with ordinary labels, the effect of negation (an explamation mark) in front
of an equality is in principle the same as
that of the embargo prefix --- but again, negations can be used \emph{within}
embargoes. This has an important use in enabling \emph{counting} in rules. For
instance, \fref{counting} specifies that a \textsf{Plate} may only be put in
the \textsf{Oven} if it contains \emph{exactly} three \textsf{Roll}s --- no
more and no less. The injectivity between the reader \textsf{Roll}s ensures
that there are no less than two of them, whereas the embargo \textsf{Roll} with
its injectivities ensures that there are no more than two.

\viewsfig{counting}{Edit and Display views of a rule with a counting constraint}

\paragraph{Example usage.}

The use of the above features is demonstrated by the following \Groove samples:
%
\begin{itemize}\noitemsep
\item \textsf{mergers}, showing the use of mergers;
\item \textsf{counting}, demonstrating the principle of counting.
\end{itemize}

\subsection{Rule Comments}
\stlabel{remark}

To document rules, \Groove{} offers the possibility to add special nodes and
edges that do not make a difference to the transformation. This is done through
the prefix ``\textsf{rem:}'' (for ``remark''), either on a node (as a
stand-alone node label) or on an edge --- just as for the prefixes we have seen
so far. In the Display view, remark nodes and edges are orange, with a yellow
background. For instance, \fref{remark} is the same rule as \fref{counting},
augmented with remarks.

\viewsfig{remark}{Edit and Display views of a rule with remark elements}

\subsection{Rule properties}
\stlabel{rule-properties}

Apart from the LHS, RHS and NAC, which are depicted graphically, a rule also
has \emph{rule properties}. These can be accessed and modified either from the
Simulator or from the Editor. The most important of these properties is the
\emph{priority} of the rule. 

\paragraph{Priorities.}

Rule priorities provide a (primitive) way to
\emph{schedule} the application of rules: as long as a high-priority rule is
enabled, no lower-priority rules can be scheduled for application.

The default rule priority is 0. Creating rules with different priorities will
change the rules overview in the Simulator: another, top level is introduced in
this view, ordering groups of rules according to their priorities.

For instance, one can introduce a high-priority rule that just tests for the
presence of an ``\textsf{Error}''-labelled node, and does not modify the
graph. Such a rule would automatically halt the transformation of a graph if
some other rule introduces such an \textsf{Error}-node.

Rule system \textsf{priorities} in the \Groove{} samples shows an example use
of priorities.

\paragraph{Confluent.}

A rule may be marked as confluent if the user is certain that it does not
make a difference in which order matches of that rule are evaluated. The effect
is only visible in one particular exploration strategy (see \stref{exploration}),
the \emph{linear confluent} exploration. In this strategy, at every state only
the first match of a confluent rule is explored.

\paragraph{Enabledness.}

A rule can be \emph{disabled}, meaning that it is never scheduled for
application. This can be very useful when developing a graph grammar, since it
makes it easy to experiment with different versions of the same rule.

\paragraph{Comment.}

The rule comment provides another way to document a rule, in addition to the
remark nodes and edges already described in \stref{remark}.

\subsection{Transition systems}
\stlabel{trans}

During the evaluation of a set of rules, \Groove{} ``under water'' builds up a
so-called transition system, in which every graph plays the role of a state,
and every rule application is interpreted as a transition. The transitions bear
the names of the rules that have been applied as labels. The precise formatting
of the transition labels can be controlled by two system properties,
\textsf{transitionBrackets} and \textsf{transitionParameters}; see
\stref{system-properties}.

\paragraph{Rule systems and grammars}

A rule system is a set of rules, possibly with some addition information such
as a control specification (see \stref{control}) and system properties (see
\stref{system-properties}). A \emph{grammar} is a rule system together with a
start graph. The default start graph, called \textsf{start}, is assumed to be
available together with the rules; other start graphs can be specified or
loaded in, depending on the circumstances.

The structure of rule names (consisting of substrings separated by periods, see
\stref{rules}) in fact imposes a hierarchy of name spaces on the rule system,
but this hierarchy does not play a role in the evaluation of a graph grammar.
In other words, the meaning of a graph grammar does not change if all the rules
are arbitrarily renamed, including renamings that change the hierarchical
structure.
