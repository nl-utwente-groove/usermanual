Control is about scheduling rule executions. It provides a much stronger
mechanism than rule priorities (see \stref{rule-properties}).

Control is specified in the form of a control program. The grammar of such programs is listed in Listing \ref{lst:control}. 

\lstset{
	basicstyle=\ttfamily\scriptsize
}

\begin{figure}
\begin{lstlisting}[label=lst:control,caption={Grammar of Control Programs}]
program
	: (function|statement)*
	;

function
	: FUNCTION IDENTIFIER '(' ')' block
	;

statement 
	: 'alap' block
	| 'while' '(' condition ')' 'do' block
	| 'do' block 'while' '(' condition ')'
	| 'until' '(' condition ')' 'do' block
	| 'try' block ('else' block)?
	| 'if' '(' condition ')' block ('else' block)?
   | 'choice' block ('or' block)*
	| expression
	;

block
	: '{' statement*  '}'
	;

condition
	: conditionliteral ('|' condition)?
	;

conditionliteral
	: 'true' | rule ;

expression	
	: expression2 ('|' expression)?
	;

expression2
    : expression_atom ('+' | '*')?
    | '#' expression_atom
    ;

expression_atom
	: rule
	| 'any'
	| 'other'
	| '(' expression ')'
	| call
	; 

call
	: IDENTIFIER '(' ')'
	;

rule
	: IDENTIFIER
	;

IDENTIFIER
	: ('a'..'z'|'A'..'Z') ('a'..'z'|'A'..'Z'|'0'..'9'|'-'|'_')*
	;
\end{lstlisting}
\end{figure}

The smallest programming elements of a control program are the names of the rules in a grammar. A control program is interpreted during exploration of the grammer. In every state, the control program decides which rules are scheduled (i.e. allowed to be applied). 

Conditional statements allow the specification of an alternative in case certain rules do not have a matching. The conditions of \textbf{if-then-else}, \textbf{while-do}, \textbf{do-while}, and \textbf{until-do} are restricted to a single rulename, \emph{true} or a choice of rules. For this choice, the condition is true when one of the options has a match. The condition is false when none of the options has a match. 

The \textbf{try-else} statement allows more complex conditions, since the condition is incorporated in the body of the first block. In this case, the condition is true when any first possible rule (according to the block) has a match. The condition is false when the block does not lead to any rule application. 
For instance, the program \texttt{try \{ a;b; \} else \{ c;d; \}} goes to the second block when rule \emph{a} does not have a match. 

The \textbf{alap} keyword stands for \emph{as long as possible}. In this case, the statement is exited when --- in a new iteration --- the block does not lead to any rule application. 

An example of control can be found in the \emph{control.gps} grammar, supplied with the samples package of groove.
